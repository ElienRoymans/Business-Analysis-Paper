\documentclass{article}
\usepackage[dutch]{babel}
\usepackage[utf8]{inputenc}
\usepackage{fancyhdr}
\usepackage{hyperref}
\usepackage{graphicx}
\usepackage{csquotes}
\usepackage{tabto}
\usepackage[style=apa,backend=biber]{biblatex}
\addbibresource{references.bib}

\pagestyle{fancy}
\fancyhf{}
\fancyfoot[C]{\thepage}

\begin{document}
    
    \begin{titlepage}
        \centering
        \vspace*{2cm}
        
        {\Huge \textbf{Business Analysis - Paper}}\\
        \vspace{1.5cm}
        
        {\Large Klas:  PBA-TIN-TI/3D}\\
        \vspace{0.5cm}
        
        {\Large Groepsnummer: 6}\\
        \vspace{0.5cm}
        
        {\Large Groepsleden: \\
            De Bie Simon\\
            De Blancq Ine\\
            Roymans Elien\\[0.15cm]
            Verschelde Flore}\\
        \vspace{1.0cm}
        
        {\Large Onderwerp casus:\\ project management in systeembeheerprojecten}\\
        
        \vfill
        
        \begin{flushright}
            \includegraphics[width=0.3\textwidth]{logo.png}
        \end{flushright}
    \end{titlepage}

    % Inhoudsopgave
    \tableofcontents
    \newpage

    % Start nummering
    \setcounter{page}{1}

    \section{Literatuur}
    \subsection{Inleiding}

    Onze paper gaat over project management in systeembeheerprojecten. 
    Project management is van uiterst belang binnen IT-projecten en het succesvol voltooien van deze projecten. 
    Het stelt teams in staat om complexe taken te plannen, structureren en te coördineren om op deze manier het doel op een efficiënte manier te bereiken.
    De meeste project management methodologieën zoals: agile, waterfall, scrum, kanban, \ldots zijn ontworpen voor softwareontwikkeling of de implementatie van software pakketten.
    Dit roept natuurlijk de vraag op of deze methodologieën ook toepasbaar zijn op systeembeheerprojecten, bijvoorbeeld op een project voor de migratie naar een nieuwe versie van OS.
    In deze paper gaan we onderzoeken of alle of toch sommige traditionele project management methodologieën toepasbaar zijn op systeembeheerprojecten.
    Het doel van deze studie is om een beter inzicht te bieden in hoe project management kan bijdragen aan het succes van systeembeheerprojecten. 

    \subsection{Projectmanagementmethodologieën in systeembeheerprojecten}

    De ISM-methode (Integrated Service Management) zorgt voor een structuur die People, Process en Product integreert om IT-dienstverlening te optimaliseren. 
    Zoals beschreven door \textcite{hoving2010ism}, combineert deze werkwijze flexibiliteit met een standaard methode met behulp van tools zoals coaching en servicedesksoftware.
    Deze methode is toepasbaar bij systeembeheerprojecten omdat hij de focus legt op structuur en eenvoud. Dit is belangrijk bij het uitvoeren van updates en migraties.
    De nadruk ligt hier op de behoeften voor methoden die buiten softwareontwikkeling ook van toepassing is binnen operationele IT-projecten. \autocite{hoving2010ism}

    Tools zoals Wrike en ClickUp werden in een vergelijkbare context onderzocht voor hun bruikbaarheid in projectbeheer.
    Uit een studie van \textcite{pasaric2022comparison} blijkt dat deze tools effectief zijn voor het combineren van plannings- en beheerfunctionaliteiten.
    Door hun flexibiliteit en integratiemogelijkheden hebben ze veel potentieel voor gebruik binnen systeembeheerprojecten terwijl ze origineel ontworpen waren voor generiek projectmanagement. \autocite{pasaric2022comparison}

    \newpage
    \subsection{Praktische toepassing van ISM en hybride methodologieën}

    Bij complexe systeembeheerprojecten, zoals cloudmigraties, is het belangrijk om een goede methodologie te kiezen. 
    ISM geeft hier een gestructureerd systeem, maar hybride aanpakken, zoals beschreven door \textcite{reiff2022hybrid}, 
    bieden extra flexibiliteit door elementen van agile en traditionele methodologieën samen te voegen.
    Voor projecten met onvoorspelbare vereisten, is dit zeer nuttig, zoals het beheren van afhankelijkheden binnen verschillende systemen of teams. \autocite{reiff2022hybrid}

    De complexiteit en implementatie vormen bij hybride methodologieën een belangrijke uitdaging.
    Dit vereist een hoog technisch niveau en goede communicatie binnen het team, maar dit biedt enorme voordelen voor de klanttevredenheid dankzij een iteratieve planningsmethode. \autocite{reiff2022hybrid}

    \subsection{Vergelijking projectmanagementtools}

    Zoals \textcite{pasaric2022comparison} concludeerden, zijn tools zoals ClickUp en Wrike het meest geschikt voor organisaties die een grote hoeveelheid functionaliteiten vereisen. 
    Tools zoals Trello die zeer gebruiksvriendelijk zijn, missen vaak belangrijke functies. 
    Sommige van deze functies zullen natuurlijk een meerwaarde hebben bij systeembeheerprojecten. \autocite{pasaric2022comparison}

    \subsection{Samenvatting}
    De ontwikkeling van Agile is ontstaan uit de nood aan flexibiliteit binnen de softwareontwikkelingsgemeenschap. 
    Personen die hiervan gebruik maken ervaren al snel de voordelen ervan.
    Sindsdien hebben Agile methoden theoretische ondersteuning gekregen voor verschillende doeleinden. \autocite{STRAY2022107058}

    \section{Hypothese}

    Traditionele project management strategieën zoals agile, waterfall en scrum zijn ontworpen voor softwareontwikkeling. 
    Wij gaan er van uit dat het mogelijk is om, met enige aanpassingen, deze methodologieën ook toe te passen op systeembeheerprojecten.
    Er zal gebruik gemaakt worden van gelijkaardige processen zoals iteratief werken.\newline
    Natuurlijk zijn er buiten gelijkenissen ook verschillen tussen software-ontwikkeling en systeembeheerprojecten. 
    Zo zullen er zeker aanpassingen nodig zijn om rekening te houden met de technische aspecten.
    Hoewel de kernprincipes van projectmanagement grotendeels hetzelfde blijven, zullen de rollen, risicoanalyses en 
    succesfactoren variëren tussen softwaregerichte en systeembeheergerichte projecten.

    
    \section{Conclusie}
    
    Wat wij uit de interviews hebben geleerd is dat het zeker mogelijk is om projectmanagement te integreren binnen systeembeheerprojecten.
    Natuurlijk staan er hier restricties aan. Zo zal het makkelijker zijn om dit toe te passen op kleinere projecten. 
    Echter is er bij softwaregerichte projecten een iteratief component en dit ontbreekt vaak bij systeembeheerprojecten. 
    Als je bijvoorbeeld twee informatica diensten van verschillende gemeentes moet samenvoegen, zal dit maar één keer moeten gebeuren.
    Bij het andere bedrijf, Corilus, proberen ze Kanban toe te passen in combinatie met Scrum, maar ook hier bevinden ze moeilijkheden met grotere projecten.
    Dit komt doordat sommige taken langer duren dan de gemiddelde sprint.
    Ze hebben echter wel rollen binnen het team, zoals een projectmanager, iemand die verantwoordelijk is voor de databases en iemand die de servers in orde brengt.
    Op basis van deze interviews is de conclusie dat er zeker voordelen zijn van het toepassen van projectmanagement methodologieën,
    maar dit geldt vooral bij de kleinere projecten. Hoe groter deze projecten hoe moeilijker deze methodologieën toepasbaar zijn.

    % Referentielijst
    \newpage
    \printbibliography
    \section{Reflectie}
    Over het algemeen was het verzamelen van informatie geen grote uitdaging. We hadden een namiddag afgesproken om zo 
    gezamenlijk naar geschikte bronnen te zoeken. Eerst verliep de zoektocht een beetje stroef omdat we niet goed wisten welke bronnen 
    geschikt waren. Uiteindelijk hebben we, hoogstwaarschijnlijk, meer bronnen gevonden dan er nodig waren. Met 
    enigszins gebruik te maken van de bronnen, en anderzijds te overleggen over de inhoud kwamen we al snel tot 10 vragen die we gebruikt hebben voor het onderzoek. 
    De bedrijven konden uitgebreid onze vragen beantwoorden. Wat de bedrijven ons wisten te zeggen kwam grotendeels overeen 
    met we ondervonden tijdens de literatuurstudie.  
    \section{Voorbereidingen interviews}

    We hadden actief gezocht naar mensen die ervaring hadden met projectmanagement in systeembeheerprojecten.
    Doordat er binnen de groep al enkele mensen waren wiens familieleden in de IT-sector werken, was het niet moeilijk om mensen te vinden die bereid waren om ons te helpen met ons onderzoek.
    Zo kwamen we op twee bedrijven uit, namelijk Corilus en bij Gemeente Deinze bij de ict dienst. \newline
    Als voorbereiding op de interviews hadden we via Proton Drive een gedeeld bestand aangemaakt waarin iedereen naar behoren zijn vragen kon neerschrijven of de andere vragen aanpassen. 
    Al snel kwamen we tot een collectie van vragen die we wilden stellen aan de bedrijven.
    De antwoorden op de vragen werden echter niet in dit bestand genoteerd, maar tijdens het interview zelf op papier vastgelegd.
    \newpage
    \subsection{Vragenlijst}
    \begin{enumerate}
        \item Wat is uw rol binnen het bedrijf, en hoe lang bent u al actief binnen deze functie?
        \item Welke relevante opleidingen en certificaten kan u al afvinken?
        \item Welke ervaring heeft u zelf met projectmanagement in systeembeheer?
        \item Kan u zelf een voorbeeld geven van een project dat u succesvol heeft kunnen beëindigen?
        \begin{itemize}
            \item Indien ja: Wat was uw rol binnen dit project?
        \end{itemize}
        \item Wat zijn de projecten waar uw team of u binnen uw team momenteel aan werkt?
        \begin{itemize}
            \item Met hoeveel mensen zit u in een team, en welke rolverdelingen zijn er?
        \end{itemize}
        \item Welke managementmethodologieën gebruiken jullie meest binnen jullie bedrijf?
        \begin{itemize}
            \item Eventuele voor- en nadelen?
        \end{itemize}
        \item Welke methodologieën gebruiken jullie voor de huidige projecten?
        \item Hoe verloopt de implementatie van deze methodologieën?
        \item Zijn deze methodes toepasbaar op elk project?
        \begin{itemize}
            \item Indien dit niet zo is: Hoe pak je dit aan?
        \end{itemize}
        \item Zijn er verschillen tussen softwaregerichte projecten en systeembeheergerichte projecten?
        \begin{itemize}
            \item Indien ja: Hoe pak je dit aan?
        \end{itemize}
    \end{enumerate}
    \newpage
    \subsection{Interviews}
    \begin{enumerate}
        \item Het interview bij Corilus was met een operations engineer. Hij heeft al zeven jaar ervaring binnen zijn 
        functie en is voornamelijk verantwoordelijk voor de server infrastructuur voor hun klanten. Van opleidingen
        heeft hij toegepaste informatica gestudeerd aan HOGENT, met als afstudeer specialisatie netwerken. Binnen
        systeembeheer is zijn ervaring als projectmanager beperkt, maar hij heeft al wel enkele kleinere projecten
        geleid. Als voorbeeld van projecten gaf hij de overnamen van een ander bedrijf. Hiervoor hebben
        zij de bestaande infrastructuur van het andere bedrijf, overgezet naar dat van hunzelf. Bij de meeste
        projecten werken zij met teams van vier: twee developers, iemand die verantwoordelijk is voor de servers en
        iemand die DNS regelt. Voor de grote projecten werken zij met meerdere teams van zes mensen.

        Op vlak van project managementmethodologieën gebruiken zij Agile PI, vroeger gebruikten ze SCRUM, maar hiervoor
        is het bedrijf te groot geworden. Nu werken zij met SCRUM elke twee weken, en hebben zij een PI vergadering elke
        tien weken. Het grootste voordeel hiervan is dat dit goed werkt voor grote bedrijven, maar een nadeel is dat
        dit proces heel tijdsintensief is. Zij ervaren echter dat dit niet toepasbaar is op elk soort project. Voor
        hun operation teams is dit niet toepasbaar, zij maken wel gebruik van KANBAN. Het probleem is vooral dat sommige
        projecten te lang duren om binnen een SCRUM van twee weken te passen. Bijvoorbeeld een update van een OS op elke
        computer duurt te lang, een blijft dan vaak aanslepen.

        \item Het interview bij Deinze was met het Diensthoofd IT van de gemeente Deinze. 
        Hij werkt daar al vijftien jaar, waarvan acht jaar als enige werknemer.
        Hij is dus verantwoordelijk voor alle IT van alle diensten van Deinze.
        Dit wilt zeggen dat hij instaat voor zowel ondersteuning, als server management, als projectmanager.
        Hij heeft een diploma toegepaste informatica - software management gehaald aan het vhti Kortrijk.
        Met een groot aantal extra certificaten, waaronder Microsoft server verdeler en kmo specialist server 2008m
        microsoft netwerken en cisco certificaat.
        Vanuit Deinze zijn er ook interne opleidingen die hij heeft gevolgd 
        waaronder ''leidinggeven'' en ''projectmanagement''.
        
        Het grootste project dat hij heeft afgewerkt is dan ook de fusie van de gemeente Nevele en Deinze,
        waarbij de twee infrastructuren en gegevens moesten samengevoegd worden, alsook het afsluiten van data.
        Op het moment zijn er bij hem 4 mensen in dienst, die elk wel hun specialiteiten hebben. 
        Hij maakt er wel een punt van om opleidingen te blijven volgen 
        zodat hij altijd kan bijblijven met zijn werknemers.
        \newpage
        Voor de grote projecten zit hij tweewekelijks samen met de klant om ervoor te zorgen dat alle wensen vervuld worden.
        Ook worden er dagelijks interne meetings gehouden om ervoor te zorgen dat iedereen up to date blijft.
        De werkvloer zelf heeft een heel horizontale hiërarchie. Doordat hij over alles kan meespreken, kan hij overal inspringen en helpen.
    \end{enumerate}
\end{document}

