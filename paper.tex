\documentclass{article}
\usepackage[dutch]{babel}
\usepackage[utf8]{inputenc}

\usepackage{graphicx}


\begin{document}
    
    \begin{titlepage}
        \centering
        \vspace*{2cm}
        
        {\Huge \textbf{Business Analysis - Paper}}\\
        \vspace{1.5cm}
        
        {\Large Klas:  PBA-TIN-TI/3D}\\
        \vspace{0.5cm}
        
        {\Large Groepsnummer: 6}\\
        \vspace{0.5cm}
        
        {\Large Groepsleden: \\
            De Bie Simon\\
            De Blancq Ine\\
            Roymans Elien\\[0.15cm]
            Verschelde Flore}\\
        \vspace{1.0cm}
        
        {\Large Onderwerp casus:\\ project management in systeembeheerprojecten}\\
        
        \vfill
        
        \begin{flushright}
            \includegraphics[width=0.3\textwidth]{logo.png}
        \end{flushright}
    \end{titlepage}
    \tableofcontents
    \section{Literatuur}
    \subsection{Inleiding}
    Onze paper gaat over project management in systeembeheerprojecten. 
    Project management is van uiterst belang binnen IT-projecten en het succesvol voltooien van deze projecten. 
    Het stelt teams in staat om complexe taken te plannen, structureren en coördineren om op deze manier het doel voor ogen op een efficiënte manier te bereiken. \newline
    De meeste project management methodologieën zoals, agile, waterfall, scrum, kanban, ... zijn ontworpen voor softwareontwikkeling of de implementatie van software pakketten.
    Dit roept natuurlijk de vraag op of deze methodologieën ook toepasbaar zijn op systeembeheerprojecten, bijvoorbeeld op een project voor de migratie naar een nieuwe versie van OS. \newline
    In deze paper gaan we onderzoeken of alle of toch sommige traditionele project management methodologieën toepasbaar zijn op systeembeheerprojecten.
    Daarnaast gaan we ook nog een vergelijking maken tussen de software-ontwikkeling en de implementatie.
    Zo gaan we zowel de gelijkenissen als de verschillen proberen te achterhalen. \newline
    Het doel van deze studie is om een beter inzicht te bieden in hoe project management kan bijdragen aan het succes van systeembeheerprojecten. 

    \subsection{Samenvatting}
    De ontwikkeling van Agile is ontstaan ​​uit de collectieve wijsheid van de softwaregemeenschap en de beoefenaars die pleitten voor de voordelen van de Agile methoden. 
    Sindsdien hebben Agile methoden theoretische ondersteuning gekregen voor enkele van die genoemde voordelen. \autocite{STRAY2022107058}
    \subsection{Hypothese}
\end{document}

